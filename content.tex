\frame {
  \frametitle{Propositions as Types}

  \begin{tabular}{rcl}
  Proposition & \(\leftrightarrow\) & Type\\
  Proof & \(\leftrightarrow\) & Program\\
  Proof simplification & \(\leftrightarrow\) & Computation
  \end{tabular}

  \begin{itemize}
    \item Syntactic \(=\).
      \AxiomC{\(\textcolor{red}{A}\)}
      \AxiomC{\(B\)}
      \BinaryInfC{\(\textcolor{blue}{A} \land B\)}
      \DisplayProof
    \item \(=\) on propositions.
    \item \(=\) as proposition.
  \end{itemize}
}

\frame {
  \frametitle{Intentional Type Theory}
  \framesubtitle{Martin-L\"of 1975}
  \begin{description}
  \item[Definitional equality \(\defeq\)]\leavevmode\\
    Definition, \textalpha\textbeta-equivalence,
    \AxiomC{\(a \defeq b\)}
    \LeftLabel{SUB}
    \UnaryInfC{\(x[a] \defeq x[b]\)}
    \DisplayProof,
    \AxiomC{\(a : A\)}
    \AxiomC{\(A \defeq B\)}
    \BinaryInfC{\(a : B\)}
    \DisplayProof.
  \item[Propositional equality (a.k.a.\ Identity) \(I(A, a, b)\)]
    \[
      r(a) : I(A, a, a)
    \]
    \begin{prooftree}
    \AxiomC{\(c[x] : C[x, x, r(x)]\)}
    \UnaryInfC{\(f(x, x, r(x)) \defeq c[x]\)}
    \end{prooftree}
    \begin{prooftree}
    \AxiomC{\(\)}
    \end{prooftree}
    \begin{align*}
    J &: ((x : A) \to C(x, x, r(x)))\\
      &\to (x y : A) (p : I(A, x, y))\\
      &\to C(x, y, p)
    \end{align*}
  \end{description}
}

\frame {
  \frametitle{Extentional Type Theory}
  \framesubtitle{Martin-L\"of 1983}

  \begin{description}
  \item[Definitional (syntactic, intentional) equality]\leavevmode\\
    Definition, \textalpha-equivalence, SUB.
  \item[Judgemental equality \(A = B\), \(a = b : A\)]\leavevmode\\
    Defined for every type, \textbeta\texteta,
    \AxiomC{\([x : A]\)}
    \noLine
    \UnaryInfC{\(f(x) = g(x) : B(x)\)}
    \LeftLabel{\textxi}
    \UnaryInfC{\(\lambda x.\ f(x) = \lambda x.\ g(x) : \Pi(x : A)B(x)\)}
    \DisplayProof,
    \AxiomC{\(a : A\)}
    \AxiomC{\(A = B\)}
    \BinaryInfC{\(a : B\)}
    \DisplayProof,
    \AxiomC{\(a = b : A\)}
    \AxiomC{\(A = B\)}
    \BinaryInfC{\(a = b : B\)}
    \DisplayProof.
  \item[Propositional equality]\leavevmode\\
    \AxiomC{\(c : I(A, a, b)\)}
    \LeftLabel{ER}
    \UnaryInfC{\(a = b : A\)}
    \DisplayProof,
    \AxiomC{\(c : I(A, a, b)\)}
    \LeftLabel{UIP}
    \UnaryInfC{\(c = r : I(A, a, b)\)}
    \DisplayProof.
  \end{description}
}

% TODO: Setoids example in Coq.

\frame {
  \frametitle{Observational Type Theory}

  \begin{description}
  \item[Definitional equality] \textalpha\textbeta\texteta.
  \item[Propositional equality]
  \begin{align*}
    (x : A) = (y : B)\\
    refl : (x : A) \to (x : A) = (x : A)
  \end{align*}
  \begin{prooftree}
  \AxiomC{\(p : S_0 = S_1\)}
  \AxiomC{\(s_0 : S_0\)}
  \LeftLabel{Coercion}
  \BinaryInfC{\(s_0[p\rangle : S_1\)}
  \end{prooftree}
  \begin{prooftree}
  \AxiomC{\(p : S_0 = S_1\)}
  \AxiomC{\(s_0 : S_0\)}
  \LeftLabel{Coherence}
  \BinaryInfC{\(s_0[\![p|\!\rangle : (s_0 : S_0) = (s_0[p\rangle : S_1)\)}
  \end{prooftree}
  \end{description}

  Result: propositional equality is extensional, definitional equality
  is decidable.
}

\frame {
  \frametitle{Computational Type Theory}

  Based on ETT. But ETT is formal type theory.

  \begin{description}
  \item[Definitional equality]\leavevmode\\
    Define the meaning of \(x = y : A\) for the types you want.
    \(x : A\) is then a shortcut for \(x = x : A\).
  \item[Propositional equality]\leavevmode\\
    The same as judgemental.
  \item[Quotient types]\leavevmode\\
    \((x, y) : A /\!/ E\). Change an equivalence relation to \(E\).
  \end{description}
}

\frame {
  \frametitle{Homotopy Type Theory}

  Definitionally like ITT but with \texteta.

  Proof relevance for equality. Think \textbeta reduction chain.
  \(a = b : A\) when there is a path from \(a\) to \(b\). As such
  no UIP (and K). OTOH have function extensionality.
}

\frame {
  \frametitle{Unified Type Theory}

  ITT + impredicative Prop.
  Equality is Leibniz: \(\forall P : A \to Prop.\ P(a) \implies P(b)\).

  % TODO: Logical frameworks.
}

\frame {
  \frametitle{ZOMBIE}

  Forget automatic \textbeta. Get equality relations from the context
  and use them.
}
