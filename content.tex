\frame {
  \frametitle{Intentional Type Theory}
  \framesubtitle{Per Martin-L\"of ``An Intuitionstic Theory of Types: Predicative Part'', 1975}
  \begin{description}
  \item[Definitional equality \(\defeq\)]\leavevmode\\
    Definition, Equivalence, \textalpha\textbeta,
    \AxiomC{\(a \defeq b\)}
    \UnaryInfC{\(x[a] \defeq x[b]\)}
    \DisplayProof,
    \AxiomC{\(a : A\)}
    \AxiomC{\(A \defeq B\)}
    \BinaryInfC{\(a : B\)}
    \DisplayProof.
  \item[Propositional equality (a.k.a.\ Identity) \(I(A, a, b)\)]
    \[
      r(a) : I(A, a, a)
    \]
    \begin{prooftree}
    \AxiomC{\(x : A\)}
    \AxiomC{\(y : A\)}
    \AxiomC{\(z : I(A, x, y)\)}
    \TrinaryInfC{\(C[x, y, z]\)}
    \AxiomC{\(c[x] : C[x, x, r(x)]\)}
    \BinaryInfC{\(f(x, x, r(x)) \defeq c[x]\)}
    \end{prooftree}
  \end{description}
  TODO: Also mention J as a modern alternative.
}

\frame {
  \frametitle{Extentional Type Theory}
  \framesubtitle{Per Martin-L\"of ``Intuitionstic Type Theory", 1980}

  \begin{description}
  \item[Definitional equality]\leavevmode\\
    Definition, Equivalence, Substituiting equals for equals, \textalpha.
    Checking that in
    \AxiomC{\(A\)}
    \AxiomC{\(B\)}
    \BinaryInfC{\(A \& B\)}
    \DisplayProof
    \(A\), \(B\) on top are the same (i.e.\ definitionally equal) to the ones on the bottom.
  \item[Judgemental equality \(A = B\), \(a = b : A\)]\leavevmode\\
    Defined for every type, \textbeta\texteta,
    \AxiomC{\([x : A]\)}
    \noLine
    \UnaryInfC{\(f(x) = g(x) : B(x)\)}
    \LeftLabel{\textxi}
    \UnaryInfC{\(\lambda x.\ f(x) = \lambda x.\ g(x) : \Pi(x : A)B(x)\)}
    \DisplayProof,
    \AxiomC{\(a \in A\)}
    \AxiomC{\(A = B\)}
    \BinaryInfC{\(a \in B\)}
    \DisplayProof,
    \AxiomC{\(a = b \in A\)}
    \AxiomC{\(A = B\)}
    \BinaryInfC{\(a = b \in B\)}
    \DisplayProof.
  \item[Propositional equality]\leavevmode\\
    \AxiomC{\(c : I(A, a, b)\)}
    \LeftLabel{ER}
    \UnaryInfC{\(a = b : A\)}
    \DisplayProof,
    \AxiomC{\(c : I(A, a, b)\)}
    \LeftLabel{UIP}
    \UnaryInfC{\(c = r : I(A, a, b)\)}
    \DisplayProof.
  \end{description}
}

% TODO: Setoids example in Coq.

\frame {
  \frametitle{Observational Type Theory}
  \framesubtitle{}

  Definitional equality: \textalpha\textbeta\texteta.

  Propositional equality is heterogeneous equality: \((x : A) = (y : B),\quad refl : (x : A) \to (x : A) = (x : A)\).
  Coercion:
  \AxiomC{\(p : S_0 = S_1\)}
  \AxiomC{\(s_0 : S_0\)}
  \BinaryInfC{\(s_0[p\rangle : S_1\)}
  \DisplayProof
  Coherence:
  \AxiomC{\(p : S_0 = S_1\)}
  \AxiomC{\(s_0 : S_0\)}
  \BinaryInfC{\(s_0[[p|\rangle : (s_0 : S_0) = (s_0[p\rangle : S_1)\)}
  \DisplayProof

  Coercion, coherence rules are defined for each and every type.

  Result: propositional equality is truly extensional.
}

% TODO: UTT Leibniz for Prop

\frame {
  \frametitle{Computational Type Theory}
  \framesubtitle{}

  Based on ETT.
  Define the meaning of \(x = y : A\) for the types you want.
  \(x : A\) is then a shortcut for \(x = x : A\). And propositional
  equality is truly identified with judgemental equality.

  Set types: \(\{x : A | p(x)}\)
  Quotient types: \((x, y) : A // E\). Change an equivalence relation to \(E\).
}

\frame {
  \frametitle{ZOMBIE}
}
