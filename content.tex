\frame {
  \frametitle{Propositions as Types}

  \begin{tabular}{rcl}
  Proposition          & \(\leftrightarrow\) & Type \\
  Proof                & \(\leftrightarrow\) & Term \\
  Proof simplification & \(\leftrightarrow\) & Computation
  \end{tabular}
}

\frame {
  \frametitle{Propositions as Types}

  \begin{tabular}{rcl}
    \(A\) & \(\leftrightarrow\) & \(A\) \\
    \(A \land B\) & \(\leftrightarrow\) & \(A \times B\) \\
    \(A \supset B\) & \(\leftrightarrow\) & \(A \to B\) \\
    \(\forall x B(x)\) & \(\leftrightarrow\) & \(\Pi (x : A) B\) \\
    \(x = y\) & \(\leftrightarrow\) & ???
  \end{tabular}
}

\frame {
  \frametitle{\(=\)}

  Logic:
  \begin{itemize}
  \item Reflexivity
    \AxiomC{\(x = x\)}
    \DisplayProof
  \item Congruence
    \AxiomC{\(x = y\)}
    \UnaryInfC{\(f(x) = f(y)\)}
    \DisplayProof
  \item Substiutivity
    \AxiomC{\(x = y\)}
    \AxiomC{\(A_x\)}
    \BinaryInfC{\(A_y\)}
    \DisplayProof
  \end{itemize}

  Type theory:
  \begin{itemize}
  \item
    \AxiomC{\(\mathcal{I} x x\)}
    \DisplayProof
  \item
    \AxiomC{\(\mathcal{I} x y\)}
    \UnaryInfC{\(\mathcal{I} (f x) (f y)\)}
    \DisplayProof
  \item
    \AxiomC{\(\mathcal{I} x y\)}
    \AxiomC{\(A_x\)}
    \BinaryInfC{\(A_y\)}
    \DisplayProof
  \end{itemize}
}

\frame {
  \frametitle{Intentional Type Theory}
  \framesubtitle{aka Martin-L\"of theory of types}
  Definitional \(f \equiv g\).

  \begin{itemize}
    \item Intentional
    \item Up to \textalpha
    \item Substiutive
  \end{itemize}
  Checking that in
  \AxiomC{\(A\)}
  \AxiomC{\(B\)}
  \BinaryInfC{\(A \& B\)}
  \DisplayProof
  \(A\), \(B\) on top are the same (i.e.\ definitionally equal) to the ones on the bottom.
}

\frame {
  \frametitle{Intentional Type Theory}
  \framesubtitle{aka Martin-L\"of theory of types}
  Judgemental \(A = B,\quad a = b \in A\)

  \begin{itemize}
    \item Holds by definition for canonical elements
    \item Up to \textbeta
    \item Substiutive
    \item
      \AxiomC{\(a \in A\)}
      \AxiomC{\(A = B\)}
      \BinaryInfC{\(a \in B\)}
      \DisplayProof
    \item
      \AxiomC{\(a = b \in A\)}
      \AxiomC{\(A = B\)}
      \BinaryInfC{\(a = b \in B\)}
      \DisplayProof
  \end{itemize}
}

\frame {
  \frametitle{Intentional Type Theory}
  \framesubtitle{aka Martin-L\"of theory of types}
  Propositional \(I(a, b, A)\)

  \begin{itemize}
    \item A type
    \item Canonical element \(r(a) : I(a, a, A)\)
    \item Eliminator \(J : \).
    \item
      \AxiomC{\(a = b \in A\)}
      \UnaryInfC{\(r(a) : I (a, b, A)\)}
      \DisplayProof
  \end{itemize}
}

\frame {
  \frametitle{Extensional Type Theory}

  \AxiomC{\(x \in I(a, b, A)\)}
  \UnaryInfC{\(a = b \in A\)}
  \DisplayProof
}

\frame {
  \frametitle{CiC}
}

\frame {
  \frametitle{UTT}
}

\frame {
  \frametitle{HoTT}
}

\frame {
  \frametitle{OTT}
}

\frame {
  \frametitle{NuPRL}
}

\frame {
  \frametitle{ZOMBIE}
}
